%% Generated by Sphinx.
\def\sphinxdocclass{report}
\documentclass[letterpaper,10pt,english]{sphinxmanual}
\ifdefined\pdfpxdimen
   \let\sphinxpxdimen\pdfpxdimen\else\newdimen\sphinxpxdimen
\fi \sphinxpxdimen=.75bp\relax
\ifdefined\pdfimageresolution
    \pdfimageresolution= \numexpr \dimexpr1in\relax/\sphinxpxdimen\relax
\fi
%% let collapsible pdf bookmarks panel have high depth per default
\PassOptionsToPackage{bookmarksdepth=5}{hyperref}

\PassOptionsToPackage{warn}{textcomp}
\usepackage[utf8]{inputenc}
\ifdefined\DeclareUnicodeCharacter
% support both utf8 and utf8x syntaxes
  \ifdefined\DeclareUnicodeCharacterAsOptional
    \def\sphinxDUC#1{\DeclareUnicodeCharacter{"#1}}
  \else
    \let\sphinxDUC\DeclareUnicodeCharacter
  \fi
  \sphinxDUC{00A0}{\nobreakspace}
  \sphinxDUC{2500}{\sphinxunichar{2500}}
  \sphinxDUC{2502}{\sphinxunichar{2502}}
  \sphinxDUC{2514}{\sphinxunichar{2514}}
  \sphinxDUC{251C}{\sphinxunichar{251C}}
  \sphinxDUC{2572}{\textbackslash}
\fi
\usepackage{cmap}
\usepackage[T1]{fontenc}
\usepackage{amsmath,amssymb,amstext}
\usepackage{babel}



\usepackage{tgtermes}
\usepackage{tgheros}
\renewcommand{\ttdefault}{txtt}



\usepackage[Bjarne]{fncychap}
\usepackage{sphinx}

\fvset{fontsize=auto}
\usepackage{geometry}


% Include hyperref last.
\usepackage{hyperref}
% Fix anchor placement for figures with captions.
\usepackage{hypcap}% it must be loaded after hyperref.
% Set up styles of URL: it should be placed after hyperref.
\urlstyle{same}


\usepackage{sphinxmessages}




\title{Lenstool}
\date{May 29, 2023}
\release{8.1}
\author{Allingham et al.\@{}}
\newcommand{\sphinxlogo}{\vbox{}}
\renewcommand{\releasename}{Release}
\makeindex
\begin{document}

\pagestyle{empty}
\sphinxmaketitle
\pagestyle{plain}
\sphinxtableofcontents
\pagestyle{normal}
\phantomsection\label{\detokenize{index::doc}}


\sphinxAtStartPar
\sphinxstylestrong{Lenstool} is a gravitational lensing software for modelling mass distribution of galaxies and clusters (strong and weak regime).

\sphinxAtStartPar
Further information may be found in the \sphinxhref{https://projets.lam.fr/projects/lenstool/wiki}{Lenstool WikiStart}.

\begin{sphinxadmonition}{note}{Note:}
\sphinxAtStartPar
For the moment, this tutorial is only focusing on the introduction of \sphinxcode{\sphinxupquote{idPIE}} potentials to perform the joint X\sphinxhyphen{}ray and lensing optimisation model of the density of dark matter and intra\sphinxhyphen{}cluster medium.
\end{sphinxadmonition}

\sphinxAtStartPar
Check out the {\hyperref[\detokenize{usage::doc}]{\sphinxcrossref{\DUrole{doc}{idPIE profile description}}}} section for further information on \sphinxcode{\sphinxupquote{idPIE}} profiles, and {\hyperref[\detokenize{tutorial::doc}]{\sphinxcrossref{\DUrole{doc}{Tutorial}}}} for a short tutorial on their usage in \sphinxcode{\sphinxupquote{lenstool}}.

\begin{sphinxadmonition}{note}{Note:}
\sphinxAtStartPar
This project is under active development. It is compatible with v8 of \sphinxcode{\sphinxupquote{lenstool}}.
\end{sphinxadmonition}


\chapter{Contents}
\label{\detokenize{index:contents}}

\section{Installation}
\label{\detokenize{installation:installation}}\label{\detokenize{installation::doc}}

\subsection{\sphinxstyleliteralintitle{\sphinxupquote{GitHub}} installation}
\label{\detokenize{installation:github-installation}}\label{\detokenize{installation:id1}}
\sphinxAtStartPar
To use the modified version of \sphinxcode{\sphinxupquote{lenstool}}, first \sphinxcode{\sphinxupquote{git clone}} it:

\begin{sphinxVerbatim}[commandchars=\\\{\}]
\PYG{g+go}{git clone https://github.com/njzifjoiez/Lenstool\PYGZus{}Bspline}
\end{sphinxVerbatim}

\sphinxAtStartPar
and switch branch:

\begin{sphinxVerbatim}[commandchars=\\\{\}]
\PYG{g+go}{git checkout idPIE\PYGZhy{}pot\PYGZhy{}joint}
\end{sphinxVerbatim}

\begin{sphinxadmonition}{note}{Note:}
\sphinxAtStartPar
TODO: HERE detail the installation if necessary
\end{sphinxadmonition}


\section{\sphinxstyleliteralintitle{\sphinxupquote{idPIE}} profile description}
\label{\detokenize{usage:idpie-profile-description}}\label{\detokenize{usage::doc}}

\subsection{\sphinxstyleliteralintitle{\sphinxupquote{dPIE}} summary}
\label{\detokenize{usage:dpie-summary}}\label{\detokenize{usage:id1}}
\sphinxAtStartPar
A summary on the \sphinxstylestrong{dual Pseudo\sphinxhyphen{}Isothermal Elliptical} matter distribution (\sphinxcode{\sphinxupquote{dPIE}}) may be found \sphinxhref{https://projets.lam.fr/projects/lenstool/wiki/PIEMD}{here}, and this type of gravitational potential is described at length in \sphinxhref{https://ui.adsabs.harvard.edu/abs/2007arXiv0710.5636E/abstract}{Elìasdòttir et al. (2007, Appendix A)}
. It is identified in \sphinxcode{\sphinxupquote{lenstool}} by id: \sphinxcode{\sphinxupquote{81}}.

\sphinxAtStartPar
Assuming we neglect ellipticity in this documentation, \sphinxcode{\sphinxupquote{dPIE}} profiles write:
\begin{equation*}
\begin{split}\rho_{\mathrm{dPIE}}(r) = \frac{\rho_0}{\left[ 1 + \left( \frac{r}{s} \right)^2 \right] \left[ 1 + \left( \frac{r}{a} \right)^2 \right]}\end{split}
\end{equation*}
\sphinxAtStartPar
where
\(\rho_0\) is the density normalisation,
\(a\) the core radius, and
\(s\) the cut radius.

\sphinxAtStartPar
A sum of \sphinxcode{\sphinxupquote{dPIE}} profiles may be assumed to represent the total matter density
\(\rho_m\)
(baryons + dark matter) in the lens:
\begin{equation*}
\begin{split}\rho_m = \sum_i \rho_{\mathrm{dPIE}, i}.\end{split}
\end{equation*}
\sphinxAtStartPar
Thus the profile of the gravitational potential
\(\Phi\) may be deduced from the \sphinxcode{\sphinxupquote{dPIE}} sum:
\begin{equation*}
\begin{split}\Phi(r) = - 4 \pi G \sum_i \int_0^r \mathrm{d}s s^{-2} \int_0^s \mathrm{d}t t^2 \rho_{\mathrm{dPIE}, i}(t).\end{split}
\end{equation*}
\sphinxAtStartPar
For one \sphinxcode{\sphinxupquote{dPIE}} profile
\(\rho_{\mathrm{dPIE}}(r)\), the potential writes:
\begin{equation*}
\begin{split}\Phi_{\mathrm{dPIE}}(r) = \frac{a^2 s^2}{a^2 - s^2} \left[ \frac{s}{r} \arctan \frac{r}{s} - \frac{a}{r} \arctan \frac{r}{a} + \frac{1}{2} \ln \left( \frac{r^2 + s^2}{r^2 + a^2} \right) \right].\end{split}
\end{equation*}

\subsection{Hydrostatic \sphinxstyleliteralintitle{\sphinxupquote{idPIE}} \protect\(n_e\protect\) ICM density profile}
\label{\detokenize{usage:hydrostatic-idpie-n-e-icm-density-profile}}
\sphinxAtStartPar
If we assume the intra\sphinxhyphen{}cluster medium (ICM) to be in hydrostatic equilibrium, we may simplify the Navier\sphinxhyphen{}Stokes equation to:
\begin{equation*}
\begin{split}\frac{\partial_r \left( n_e T_e \right)}{n_e} = \frac{\mu_g m_a}{k_B} \partial_r \Phi,\end{split}
\end{equation*}
\sphinxAtStartPar
where
\(n_e\) is the ICM electron number volume density,
\(T_e\) the ICM electron temperature,
\(\mu_g \approx 0.60\) the mean molecular weight of the ICM gas,
\(m_a \approx 1.66 \times 10^{-27}\) kg the unified atomic mass, and
\(k_B\) the Boltzmann constant.

\sphinxAtStartPar
Assuming the temperature
\(T_e\) to be a function of the electronic density, we can integrate this expression to:
\begin{equation*}
\begin{split}\mathcal{J}_z (n_e) = \int_0^{n_e} \mathrm{d} n \frac{n T_e (n)}{n} = \frac{\mu_g m_a}{k_B} \Phi (r),\end{split}
\end{equation*}
\sphinxAtStartPar
where
\(\mathcal{J}_z\) is a bijection, as long as the radial density profile
\(\rho_m\) is a sum of \sphinxcode{\sphinxupquote{dPIE}} potentials.
Using a self\sphinxhyphen{}similar polytropic temperature profile, the
\(\mathcal{J}_z\) integral only depends on redshift
\(z\).
Bijections being invertible functions, we may revert the previous equation, thus yielding the \sphinxcode{\sphinxupquote{idPIE}} density profile:
\begin{equation*}
\begin{split}n_e = \mathcal{J}^{-1}_z  \left( \frac{\mu_g m_a}{k_B} \Phi (r) \right).\end{split}
\end{equation*}

\subsection{ICM profile optimisation with \sphinxstyleliteralintitle{\sphinxupquote{idPIE}} profile}
\label{\detokenize{usage:icm-profile-optimisation-with-idpie-profile}}
\sphinxAtStartPar
Given the
\(n_e\) ICM electron density, we can compute
\(S_X\), the X\sphinxhyphen{}ray surface brightness:
\begin{equation*}
\begin{split}S_X (x, y, \Delta E) = \frac{1}{4 \pi (1 + z)^4} \frac{\mu_e}{\mu_H} \int_{\mathrm{l.o.s.}} n_e^2 (x, y, l) \Lambda (\Delta E (1 + z), T_e, Z) \mathrm{d}l,\end{split}
\end{equation*}
\sphinxAtStartPar
where
\(\Delta E\) is the observed energy band,
\(z\) is the cosmological redshift of the lens,
\(\mu_e\) and
\(\mu_H\) are respectively the mean molecular weight of electron and hydrogen, and
\(\Lambda\) is the normalised cooling function (in
\(\mathrm{J.m}^3.\mathrm{s}^{-1}\)) for an ICM electron temperature
\(T_e\) and metallicity
\(Z\).
Here, we assume the metallicity to be constant throughout the cluster
\(Z = 0.3 Z_{\odot}\).

\sphinxAtStartPar
Once the model surface brightness map computed, it is compared to observations of \sphinxstyleemphasis{Chandra} or \sphinxstyleemphasis{XMM\sphinxhyphen{}Newton} X\sphinxhyphen{}ray satellites.

\begin{sphinxadmonition}{note}{Note:}
\sphinxAtStartPar
TODO: See section on statistics for more details.
\end{sphinxadmonition}


\section{Tutorial}
\label{\detokenize{tutorial:tutorial}}\label{\detokenize{tutorial::doc}}

\subsection{Use \sphinxstyleliteralintitle{\sphinxupquote{idPIE}} X\sphinxhyphen{}ray profiles.}
\label{\detokenize{tutorial:use-idpie-x-ray-profiles}}\label{\detokenize{tutorial:idpie-tutorial}}
\sphinxAtStartPar
To use \sphinxcode{\sphinxupquote{idPIE}} profiles, one must choose which \sphinxcode{\sphinxupquote{dPIE}} profiles are considered to trace the X\sphinxhyphen{}ray signal.
The \sphinxcode{\sphinxupquote{idPIE}} profiles use the same parameters as the \sphinxcode{\sphinxupquote{dPIE}} profiles, but convert them into their corresponding hydrostatic ICM density, and computes the expected X\sphinxhyphen{}ray signal. The joint optimisation of selected profiles yields additional constraints.
In practice, \sphinxcode{\sphinxupquote{dPIE}} profiles (id:\sphinxcode{\sphinxupquote{81}}) are co\sphinxhyphen{}optimised with X\sphinxhyphen{}ray using \sphinxcode{\sphinxupquote{idPIE}} profiles if keyword \sphinxcode{\sphinxupquote{X\sphinxhyphen{}ray   2}} is added to the profile script.

\sphinxAtStartPar
For instance:

\begin{sphinxVerbatim}[commandchars=\\\{\}]
\PYG{g+go}{potential O1}
\PYG{g+go}{        profile          81}
\PYG{g+go}{        X\PYGZhy{}ray            2}
\PYG{g+go}{        x\PYGZus{}centre         0.}
\PYG{g+go}{        y\PYGZus{}centre         0.}
\PYG{g+go}{        ellipticity      0.5}
\PYG{g+go}{        angle\PYGZus{}pos        0.}
\PYG{g+go}{        core\PYGZus{}radius\PYGZus{}kpc  100}
\PYG{g+go}{        cut\PYGZus{}radius\PYGZus{}kpc   2500.}
\PYG{g+go}{        v\PYGZus{}disp           1000.}
\PYG{g+go}{        z\PYGZus{}lens           0.3}
\PYG{g+go}{        end}
\PYG{g+go}{limit O1}
\PYG{g+go}{        x\PYGZus{}centre         1 \PYGZhy{}10. 5. 0.01}
\PYG{g+go}{        cut\PYGZus{}radius\PYGZus{}kpc   1 500. 10000. 100.}
\PYG{g+go}{        end}
\end{sphinxVerbatim}


\section{Potfile}
\label{\detokenize{potfile:potfile}}\label{\detokenize{potfile::doc}}

\subsection{Use \sphinxstyleliteralintitle{\sphinxupquote{potfile}} keyword for file of potentials optimised together (following a scaling relationship).}
\label{\detokenize{potfile:use-potfile-keyword-for-file-of-potentials-optimised-together-following-a-scaling-relationship}}\label{\detokenize{potfile:id1}}
\sphinxAtStartPar
For instance:

\begin{sphinxVerbatim}[commandchars=\\\{\}]
\PYG{g+go}{potfile 1}
\PYG{g+go}{        filein        9 potfile.cat}
\PYG{g+go}{        zlens         0.3}
\PYG{g+go}{        type          81}
\PYG{g+go}{        mag0          20.}
\PYG{g+go}{        corekpc       0.15}
\PYG{g+go}{        sigma         3 190. 5.}
\PYG{g+go}{        cutkpc        3 10. 3.}
\PYG{g+go}{        slope\PYGZus{}FJ      3 1. 0.1}
\PYG{g+go}{        Zero\PYGZus{}point\PYGZus{}FP 3 \PYGZhy{}0.6 0.03}
\PYG{g+go}{        slope\PYGZus{}SB      3 0.30 0.02}
\PYG{g+go}{        Factor\PYGZus{}Re     3 2. 0.35}
\PYG{g+go}{        vdscatter     0 0. 0.}
\PYG{g+go}{        rcutscatter   0 0. 0.}
\PYG{g+go}{        pivot\PYGZus{}sigma   2.}
\PYG{g+go}{        pivot\PYGZus{}mu      20.}
\PYG{g+go}{        end}
\end{sphinxVerbatim}



\renewcommand{\indexname}{Index}
\printindex
\end{document}